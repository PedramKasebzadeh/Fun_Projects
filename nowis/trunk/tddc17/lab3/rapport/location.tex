\subsection{Locations}

We will try to increase the number of locations.
The basic problem is composed of 4 cities with 4 truck, 1 train, 1 airplane. 2 airports (in city1 and city2) and 3 train station (city2, city3 and city4).

  \subsubsection{FF planer} \label{location_ff}

\begin{tabular}{|c|c|c|}
  \hline
  number of locations & total time & evaluated states (max depth) \\ \hline
  9  & 17.56s & 1823 (13) \\ \hline
  14 & 29.40s & 1823 (13) \\ \hline
  20 & 50.80s & 1823 (13) \\ \hline
\end{tabular}

This planer prefer to use truck to move a packet from a city to another instead of using faster transporter like train or airplane.

But as we can see, increasing the number of location don't increase the number of evaluated states, it only increase the time of searching. That means that it took more time to evaluate a state. This is understandable because at each state, if there is more location (even dummies one), more actions could be possibles. The planner have to compute action likes drive(truck location) for each locations.

For each entry, ff has returned the same plan. Which is the behavior we expected because the new locations are not in the goal.

  \subsubsection{IPP planer} \label{location_ipp}
\begin{tabular}{|c|c|c|}
  \hline
  number of locations & total time & actions tried \\ \hline
   9  & 5.72s  & 64345 \\ \hline
   14 & 57.33s & 363388 \\ \hline
   20 & 362.28s & 1067026 \\ \hline
\end{tabular}

Instead of FF planer, IPP choose to use both train and truck to move packet between cities.

IPP seams to take care of dummies locations. Maybe, it will try to bring packet to these locations to see if it can reach the goal that way.

Increasing the number of location seams to increase polynomially the number of explored states, and so increase polynomially the total time of exploration.

However, IPP seams to reach a first goal node extremely quickly.

